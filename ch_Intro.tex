\chapter{Introduction}\label{ch:intro}
The Large Hadron Collider (LHC) is the world's largest and most powerful particle collider. It was built by the European Organization for Nuclear Research (CERN) between 1998 and 2008. From the early 2010 up until today it has produced several proton run yielding a great amount of data gathered by the four main experiments (CMS, ATLAS, ALICE and LHCB). The data allowed for the publication of hundreds of articles and between those one of the most important: the experimental confirmation of the Higgs Boson existence.\\ The current configuration of LHC will remain substantially unvaried until 2023 when, during the Long Shutdown 3 (LS3), several upgrades of the accelerator, will allow a significant increase of the instantaneous luminosity delivered. The upgraded version of the collider is referred as High Luminosity Large Hadron Collider (HL-LHC).

\section{HL-LHC}
It is expected that by 2023 the quadrupoles magnet that focus the proton beans of LHC for the two biggest experiment (CMS and ATLAS) will be close to the end of their life cycle due to radiation exposure. The substitution of those quadrupoles and the addition of new upgrades to optimize the bunch overlap at the interaction region, will result in a significant increase of the delivered luminosity.
The proposed operating scenario is to have $5 \times 10^{34} \unit{cm^{-2}s^{-1}$ of instantaneous luminosity, roughly five times more than the current one.\\
As a direct consequence of the increase in luminosity a high number of events happening during the same bunch crossing is expected, this phenomena is often referred to as \textit{pileup}. 

\section{HGCAL description}

\section{Clustering overview}\label{sec:hgcal_clustering}

\section{HLT for HGCAL needs}

\section{Computation challanges}
